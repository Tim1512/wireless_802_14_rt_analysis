\subsection{Rodzina standardów komunikacyjnych IEEE 802.15 :} 

\par IEEE 802.15 jest nazwą roboczej grupy standardów opracowanych przez "Institute of Electrical and Electronics Engineers" dotyczących dobrych praktyk oraz budowy aplikacji typu bezprzewodowa komunikacja radiowa na nielicencjonowanym paśmie częstotliwości którym w większości miejsc na świecie jest 2.4GHz. 
Standardów wchodzących w skład IEEE 802.15 jest siedem grup i wszystkie one dotyczą sieci bezprzewodowych typu WPAN (wireless personal area network).
WPAN jest to standard sieci bezprzewodowych zazwyczaj o niewielkim zasięgu służące do prostej komunikacji między urządzeniami. 
\par W sieciach tego typu dochodzi do komunikacji między dwoma lub większą ilością urządzeń i najczęściej istnieje podział na urządzenia podrzędne i nadrzędne.
Ze względu na budowę WPAN są najczęściej opisywane za pomocą modelu matematycznego jakim jest graf, z tego powodu urządzenia wewnątrz sieci są również nazywane węzłami sieci a połączenia między członkami sieci krawędziami.
\par WPAN mają wiele możliwości konfiguracji ze względu na wzajemne położenie lub funkcje urządzeń należących do sieci i jest nazywana potocznie architekturą sieci i określa ona zarówno wzajemne zależności funkcjonalne jak i fizyczne połorzenie węzłów sieci. 
Struktury wzajemnych połączeń tworzące architekturę mogą być stałe lub tworzone na bierząco dla określonej tymczasowej potrzeby. Dzięki temu sieci mają możliwość zmiany struktury sieci w ciągu kilki sekund.
Konkretnymi technologiami umożliwiającymi tworzenie sieci WPAN są Bluetooth, ZigBee, Z-Wave, Wireless USB, Ultra Wideband, IrDA, HomeRF/INSTEON i inne.
\\
\\
{\centering 
 Grupy standardów wchodzące w skład IEEE 802.15:
}
\\
\par - Group 1. WPAN/Bluetooth : najbardziej znana grupa bazująca na technologii Bluetooth definiuje warstwę fizyczną (PHY) oraz warstwę kontroli dostępu (MAC) Rozwiązania te powstały z myślą o przenośnych i mobilnych urządzeniach. Standard 802.15.1 ostatnio był aktualizowany w 2002 i 2005 roku.
\\
\par- Group 2. Standard IEEE 802.15.2 z 2003 roku dotyczy zalecanych praktyk dla technologii informatycznych w lokalnych i miejskich sieciach oraz współistnienia bezprzewodowych sieci i innymi urządzeniami bezprzewodowymi pracujących na nielicencjonowanym paśmie częstotliwości.
\\
\par- Group 3. High Rate WPAN : standard IEEE 802.15.3-2003 dotyczy dobrych praktyk i zaleceń odnośnie warstw fizycznej (PHY) oraz dostępu do medium (MAC) dla sieci z wysokimi prędkościami transferu tj. 11-55 Mbit/s. Grupa ta powstała z myślą o urządzeniach HDTV, "video on demand" czy "real time streaming" w których duże ilości danych (np. film) są transmitowane bezprzewodowo do odbiornika (np FullHD TV).
\\
\par- Group 4. Low Rate WPAN : standard traktujący o sieciach o niskiej szybkości transferu danych ale bardzo wysokiej energo-oszczędności głównie dla urządzeń o zasilaniu bateryjnym. 802.15.4 definiuje warstwę fizyczną (PHY) oraz warstwę łącza danych (data-link) czyli warstwy pierwzą i drugą modelu OSI. Pierwsza wersja standardu zostałą wydana w 2003 roku i obejmowała również wiele standaryzowanych oraz przemysłowych architektur sieci dla konkretnych protokołów korzystających z tego standardu takich jak:  ZigBee, 6LoWPAN, WirelessHART.
802.15.4 posiada również kilka rewizji (od a do g) większość z nich (a-e)była sporządzona w celu implementacji warstwy fizycznej na innych częstotliwościach niż bazowe 2.4GHz. Konieczność ta pojawiła się głównie ze względu na specyficzne przepisy w krajach takich jak Japonia, Chiny czy Korea.
\\
\par- Group 5. Mesh Networking : standard dostarczający architekturalne praktyki oraz zalecenia dla sieci WPAN w których skład wchodziły współpracujące ze sobą i zintegrowane High-Rate (IEEE 802.15.3) i Low-Rate WPAN (IEEE 802.15.4). 
\\
\par- Group 6. Body Area Networks : standard powstały w roku 2011 o nazwie BAN - Body Area Network. Są to sieci które cechują się wysoka energooszczędnością, krótkim zasięgiem działania o zastosowaniu na/w/wokół ludzkiego ciała (oraz nie tylko ludzkiego) które mają służyć do różnych zastosowań, w tym elektroniki medycznej, konsumenckiej i tej z branży osobistej rozrywki.
\\
\par- Group 7. Visible light communication : Grupa standardów powstałą w 2011 roku, IEEE 802.15.7 dotyczy implementacji warstw PHY i MAC dla komunikacji za pomocą światła widzialnego


\clearpage 