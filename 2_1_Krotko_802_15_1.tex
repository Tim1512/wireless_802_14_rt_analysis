\subsubsection{Wprowadzenie do standardu 802.15.1 Bluetooth}

\par
\tab Standard Bluetooth jak wszystkie z rodziny IEEE.802.15 działa na częstotliwości 2.4 GHz. Wywodzi się on ze standardu 802.15.1 ale z czasem powstała specjalna grupa zajmująca się jedynie technologią Bluetooth i sam standard 802.15.1 przestał być rozwijany kosztem samego protokołu Bluetooth. Obecnie standard ten został wypuszczony w kilku wersjach: 1.0, 1.1, 1.2, 2.0,2.1, 3.0, 4.0, 4.1\\
Widmo sygnału rozciąga się od 2400 do 2483.5 MHz i zawiera w sobie 79 kanałów transmisyjnych każdy o częstotliwości 1MHz. (Standard 4.0 korzysta z 40 kanałów o szerokości 2MHz). Początkowo korzystał on jedynie z Modulacji GFSK (Gaussian frequency-shift keying) ale z czasem zaczęto stosować (wersje 2.0+) DQPSK  (Differential Quaternary Phase Shift Keying) i 8DPSK.

\clearpage 