
\section*{spis treści}

\begin{itemize}
            \item [1] Wstęp
            \item [2] Wprowadzenie teoretyczne
            \begin{itemize}
            		\item [2.1] Sieci bezprzewodowe krótkiego zasięgu
            		\item [2.2] Standard IEEE 802.15
            		\begin{itemize}
            			\item [2.2.1] Standard 802.15.1 /BLE
        				\item [2.2.2] Standard 802.15.4
        				\item [2.2.3] Zigbee
            		\end{itemize}
            		\item [2.3] Budowa protokołu ZigBee
            		\item [2.4] Budowa protokołu Bluetooth Low Energy
            		\item [2.5] Bezprzewodowe sieci sensoryczne WSN
            		\item [2.6] Omówienie zaganień bezpieczeństwa danych w WSN
            		\begin{itemize}
            			\item [2.6.1] Bezpieczeństwo i poufność danych w systemach wireless
            			\item [2.6.2] Zigbee Security
            			\item [2.6.3] Bluetooth Low Energy Security
            		\end{itemize}
            		\item [2.7] Systemy embedded w skrócie
            		\begin{itemize}
            			\item [2.7.1] Natywne systemy operacyjne
            			\item [2.7.2] Systemy typu RTOS
        				\item [2.7.3] Systemy Embedded Linux
            		\end{itemize}
            \end{itemize}
            \item [3] Implementacja heterogenicznego systemu WSN/RT
            \begin{itemize}
            	\item [3.1] Samodzielny System Czasu Rzeczywistego
            	\begin{itemize}
            			\item [3.1.1] Podstawowe pojęcia 
            			\item [3.1.2] Sygnały We/Wy
            			\item [3.1.3] Charakterystyka ogólna systemu-RT
            	\end{itemize}
            	\item [3.2] Heterogeniczny system WSN/RT
            	\begin{itemize}
            			\item [3.2.1] Budowa systemu
            			\item [3.2.2] Przedstawienie badanego problemu
        				\item [3.2.3] Opis domeny zagadnienia
        				\item [3.2.4] Analiza części składowych systemu
        				\item [3.2.5] Proponowane rozwiązanie
            	\end{itemize}	
            \end{itemize}
            \item [4] Przedstawienie wyników badań i ich omówienie
            \begin{itemize}
            	\item [4.1] Polityka Bezpieczeństwa informacji
            	\item [4.2] Analiza czasowa procesu RT 
            	\item [4.3] Analiza kwestii bezpieczeństwa i przedstawienie znanych zagrożeń
            	\item [4.4] Podsumowanie wyników i wnioski
            \end{itemize}
            \item [5] Bibliografia
        \end{itemize}


\clearpage 