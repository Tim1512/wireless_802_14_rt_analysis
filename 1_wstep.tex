\section{Wstęp:} 

\par 
\tab Od zawsze postęp technologiczny ułatwiał przeciętnym ludziom życie czyniąc przedmioty codziennego użycia łatwiejszymi w obsłudze, wygodniejszymi i bardziej przyjaznymi użytkownikowi. Również zawsze z postępem technologii inżynierowie stawali przed nowymi wyzwaniami i problemami które musieli rozwiązać aby stworzyć projekt. Nie inaczej było z bezprzewodową transmisją danych która sukcesywnie w ostatnich latach wypiera przewodowe rozwiązania. Dzięki temu użytkownicy mogą zaoszczędzić miejsce ograniczając liczbę kabli które przysłowiowo walają się po biurku, ale również i dużo mniej materiału takiego jak miedź się zużywa na łączenie wszystkich elektronicznych komponentów.
\\
\par Powstało kilka standardów transmisji radiowej opartej o standard IEEE.802, do najbardziej znanych przedstawicieli tej grupy należy transmisja oparta o WiFi czy Bluetooth. Oprócz nich istnieje jeszcze kilka standardów należących do rodziny IEEE.802.15 mniej znanyh przeciętnemu użytkownikowi jak ZigBee czy BLE (Bluetooth Low Energy implementujący standard Bluetooth 4.0)
\\
\par Niniejsza praca traktuje o systemach czasu rzeczywistego opartych o architekture bezprzewodowych sieci sensorycznych w których wykorzystywany do komunikacji są protokoły należące do rodziny IEEE.802.15 takie jak ZigBee czy Bluetooth Low Energy ale również zostaną omówione inne możliwe rozwiązania.
\\
\par Celem pracy jest implementacja systemu czasu rzeczywistego jako wzorcowego i modularnego rozwiązania, oraz zbadanie jego bezpieczeństwa za pomocą metryk spotykanych przy audytach czy badaniu komercyjnych systemów informatycznych. \\
\tab Wartość merytoryczna samej pracy nie jest oderwana od rzeczywistości ponieważ tworzone rozwiązanie odzwierciedla przykład zastosowania bezprzewodowych sieci sensorycznych w dziedzinie przemysłu jakim jest rynek urządzeń elektromedycznych, a więc rozważane kwestie są na tyle poważne i sprawdzone aby mogły być z powodzeniem użyte dla rzeczywistych aparatów medycznych. \\
\tab W zakres pracy wchodzi również zbudowanie i implementacja heterogenicznej bezprzewodowej sieci sensorycznej poczynając od stworzenia warstwy sprzętowej analogowej oraz cyfrowej, i napisaniem oprogramowania które to spełni wymagania dotyczące funkcjonalności układu.



\clearpage