\section{Przedstawienie wyników badań i ich omówienie}

\subsection{Polityka Bezpieczeństwa informacji}

W celu określenia czy dane rozwiązanie można nazwać bezpiecznym czy nie trzeba posłóżyć się pewną metryką. Sprawdzoną metryką jest tzw: \textit{Triada CIA - Confidentiality, integrity and availability}, pozwala ona na odniesienie dowolnego systemu informatycznego do modelu CIA poprzez analizę systemu/zasobów informatycznych w trzech wymiarach: poufności, integralności i dostępności.

\begin{enumerate}
	\item Poufność (\textit{Confidentiality}): Każdy system powinien mieć określone grupy odbiorców którzy mogą mieć dostęp do określonych danych. Poufność jest to cecha systemu informatycznego gwarantująca, że dane w systemie będą jedynie dostępne dla określonych osób/podmiotów i osoby trzecie (tzw. intruzi) nie będą w stanie poznaćtych danych. Ta włąściwość systemów informatycznych jest najczęściej zapewniana przez kryptografię.
	\item Integralność (\textit{Integrity}): Systemy komunikacyjne jak i również przechowujące dane muszą byś w stanie zapewnić integralność dostarczanych danych czyli innymi słowy być odporne na zgubienie danych czy dostarczenie uszkodzonych/niepełnych/przekłamnych danych. W systemach służących do komunikacji ta cecha jest najczęściej realizowana za pomocą podziału danych na pakiety i sprawdzaniu: czy pakiet jest nie uszkodzony oraz czy zależności między pakietami są właściwe (czy nie ma zgubionych lub nadmiarowych pakietów danych)
	\item Dostępność(\textit{Availability})): Główną funkcją wszelakich systemów informatycznych jest funkcja dostarczania danych do odbiorcy. Dostępność gwarantuje nam, że po wysłaniu prawidłowego żądania dostępu do danych otrzymamy je w skończonym-określonym czasie.
\end{enumerate}

\par
\tab Ponieważ opisany wcześniej System RT jest systemem krytycznym ze względu na funkcję \textit{(przyp. Sterowanie Laserem Medycznym przeznaczonym do wykonywania zabiegów na ludziach)} \\
Aby zbadać czy zbudowany System RT jest \textit{"Bezpieczny"} zmierzymy go w wymiarach: Poufności, Integralności oraz Dostępności danych. Następnie na podstawie tych wyników spróbujemy określić czy spełnia on wymogi bezpieczeństwa modelu CIA. \\
Aby określić co tak naprawdę musi zostać zbadane należy przyporządkować do właściwości C-I-A określone moduły, właściwości danych:
\begin{enumerate}
	\item Poufność - Wiąże się bezpośrednio z szyfrowaniem danych wysyłanych w pakietach oraz z kwestiami takimi jak ustalenie bezpiecznej sesji czy działaniem mechanizmów komunikacyjnych protokołu. W celu ustalenia czy system spełnia dane wymogi poufności w dalszej części zostanie przeprowadzona analiza bezpieczeństwa dla rozwiązania opartego o protokół komunikacyjny ZigBee oraz BLE.
	\item Integralność - Za tę właściwość odpowiadają warstwy fizyczne protokołu (W wypadku Zigbee - protokół 802.15.4 natomiast dla BLE 802.15.1) ponieważ ta właściwość jest określona przez sam protokół i zapewniona za pomocą stosownych mechanizmów takich jak np. CRC dla każdego pakietu czy stos protokołu, nie będzie ona w ramach niniejszej pracy szczegółowo badana.
	\item Dostępność - Zależy bezpośrednio od tego jak szybko jest w stanie odbiornik odpowiedzieć na żądanie przychodzące z nadajnika. Jest to również jedna z cech systemów real-time. Aby sprawdzić czy system spełnia wymogi dostępności w kolejnej części zostaną przedstawione wyniki eksperymentu oraz ich analiza.
\end{enumerate}

W dalszej części zostaną przedstawione wyniki "Analizy czasowej systemu RT" oraz "Analiza podatności i zagrożeń dla protokołów ZigBee i BLE". Następnie w odniesieniu do modelu CIA zostanie określone czy dany System-RT jest \textit{Bezpieczny} czy nie, a ponieważ niniejszy system RT jest systemem wzorcowym i modularnym, na jego podstawie będzie można zadecydować czy inny system opierający się o protokół komunikacji BLE/ZigBee może być bezpiecznym, ponieważ analiza będzie ogólna.

\clearpage